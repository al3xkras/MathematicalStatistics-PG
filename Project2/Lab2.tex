\documentclass[11pt]{article}

    \usepackage[breakable]{tcolorbox}
    \usepackage{parskip} % Stop auto-indenting (to mimic markdown behaviour)
    

    % Basic figure setup, for now with no caption control since it's done
    % automatically by Pandoc (which extracts ![](path) syntax from Markdown).
    \usepackage{graphicx}
    % Maintain compatibility with old templates. Remove in nbconvert 6.0
    \let\Oldincludegraphics\includegraphics
    % Ensure that by default, figures have no caption (until we provide a
    % proper Figure object with a Caption API and a way to capture that
    % in the conversion process - todo).
    \usepackage{caption}
    \DeclareCaptionFormat{nocaption}{}
    \captionsetup{format=nocaption,aboveskip=0pt,belowskip=0pt}

    \usepackage{float}
    \floatplacement{figure}{H} % forces figures to be placed at the correct location
    \usepackage{xcolor} % Allow colors to be defined
    \usepackage{enumerate} % Needed for markdown enumerations to work
    \usepackage{geometry} % Used to adjust the document margins
    \usepackage{amsmath} % Equations
    \usepackage{amssymb} % Equations
    \usepackage{textcomp} % defines textquotesingle
    % Hack from http://tex.stackexchange.com/a/47451/13684:
    \AtBeginDocument{%
        \def\PYZsq{\textquotesingle}% Upright quotes in Pygmentized code
    }
    \usepackage{upquote} % Upright quotes for verbatim code
    \usepackage{eurosym} % defines \euro

    \usepackage{iftex}
    \ifPDFTeX
        \usepackage[T1]{fontenc}
        \IfFileExists{alphabeta.sty}{
              \usepackage{alphabeta}
          }{
              \usepackage[mathletters]{ucs}
              \usepackage[utf8x]{inputenc}
          }
    \else
        \usepackage{fontspec}
        \usepackage{unicode-math}
    \fi

    \usepackage{fancyvrb} % verbatim replacement that allows latex
    \usepackage{grffile} % extends the file name processing of package graphics
                         % to support a larger range
    \makeatletter % fix for old versions of grffile with XeLaTeX
    \@ifpackagelater{grffile}{2019/11/01}
    {
      % Do nothing on new versions
    }
    {
      \def\Gread@@xetex#1{%
        \IfFileExists{"\Gin@base".bb}%
        {\Gread@eps{\Gin@base.bb}}%
        {\Gread@@xetex@aux#1}%
      }
    }
    \makeatother
    \usepackage[Export]{adjustbox} % Used to constrain images to a maximum size
    \adjustboxset{max size={0.9\linewidth}{0.9\paperheight}}

    % The hyperref package gives us a pdf with properly built
    % internal navigation ('pdf bookmarks' for the table of contents,
    % internal cross-reference links, web links for URLs, etc.)
    \usepackage{hyperref}
    % The default LaTeX title has an obnoxious amount of whitespace. By default,
    % titling removes some of it. It also provides customization options.
    \usepackage{titling}
    \usepackage{longtable} % longtable support required by pandoc >1.10
    \usepackage{booktabs}  % table support for pandoc > 1.12.2
    \usepackage{array}     % table support for pandoc >= 2.11.3
    \usepackage{calc}      % table minipage width calculation for pandoc >= 2.11.1
    \usepackage[inline]{enumitem} % IRkernel/repr support (it uses the enumerate* environment)
    \usepackage[normalem]{ulem} % ulem is needed to support strikethroughs (\sout)
                                % normalem makes italics be italics, not underlines
    \usepackage{mathrsfs}
    

    
    % Colors for the hyperref package
    \definecolor{urlcolor}{rgb}{0,.145,.698}
    \definecolor{linkcolor}{rgb}{.71,0.21,0.01}
    \definecolor{citecolor}{rgb}{.12,.54,.11}

    % ANSI colors
    \definecolor{ansi-black}{HTML}{3E424D}
    \definecolor{ansi-black-intense}{HTML}{282C36}
    \definecolor{ansi-red}{HTML}{E75C58}
    \definecolor{ansi-red-intense}{HTML}{B22B31}
    \definecolor{ansi-green}{HTML}{00A250}
    \definecolor{ansi-green-intense}{HTML}{007427}
    \definecolor{ansi-yellow}{HTML}{DDB62B}
    \definecolor{ansi-yellow-intense}{HTML}{B27D12}
    \definecolor{ansi-blue}{HTML}{208FFB}
    \definecolor{ansi-blue-intense}{HTML}{0065CA}
    \definecolor{ansi-magenta}{HTML}{D160C4}
    \definecolor{ansi-magenta-intense}{HTML}{A03196}
    \definecolor{ansi-cyan}{HTML}{60C6C8}
    \definecolor{ansi-cyan-intense}{HTML}{258F8F}
    \definecolor{ansi-white}{HTML}{C5C1B4}
    \definecolor{ansi-white-intense}{HTML}{A1A6B2}
    \definecolor{ansi-default-inverse-fg}{HTML}{FFFFFF}
    \definecolor{ansi-default-inverse-bg}{HTML}{000000}

    % common color for the border for error outputs.
    \definecolor{outerrorbackground}{HTML}{FFDFDF}

    % commands and environments needed by pandoc snippets
    % extracted from the output of `pandoc -s`
    \providecommand{\tightlist}{%
      \setlength{\itemsep}{0pt}\setlength{\parskip}{0pt}}
    \DefineVerbatimEnvironment{Highlighting}{Verbatim}{commandchars=\\\{\}}
    % Add ',fontsize=\small' for more characters per line
    \newenvironment{Shaded}{}{}
    \newcommand{\KeywordTok}[1]{\textcolor[rgb]{0.00,0.44,0.13}{\textbf{{#1}}}}
    \newcommand{\DataTypeTok}[1]{\textcolor[rgb]{0.56,0.13,0.00}{{#1}}}
    \newcommand{\DecValTok}[1]{\textcolor[rgb]{0.25,0.63,0.44}{{#1}}}
    \newcommand{\BaseNTok}[1]{\textcolor[rgb]{0.25,0.63,0.44}{{#1}}}
    \newcommand{\FloatTok}[1]{\textcolor[rgb]{0.25,0.63,0.44}{{#1}}}
    \newcommand{\CharTok}[1]{\textcolor[rgb]{0.25,0.44,0.63}{{#1}}}
    \newcommand{\StringTok}[1]{\textcolor[rgb]{0.25,0.44,0.63}{{#1}}}
    \newcommand{\CommentTok}[1]{\textcolor[rgb]{0.38,0.63,0.69}{\textit{{#1}}}}
    \newcommand{\OtherTok}[1]{\textcolor[rgb]{0.00,0.44,0.13}{{#1}}}
    \newcommand{\AlertTok}[1]{\textcolor[rgb]{1.00,0.00,0.00}{\textbf{{#1}}}}
    \newcommand{\FunctionTok}[1]{\textcolor[rgb]{0.02,0.16,0.49}{{#1}}}
    \newcommand{\RegionMarkerTok}[1]{{#1}}
    \newcommand{\ErrorTok}[1]{\textcolor[rgb]{1.00,0.00,0.00}{\textbf{{#1}}}}
    \newcommand{\NormalTok}[1]{{#1}}

    % Additional commands for more recent versions of Pandoc
    \newcommand{\ConstantTok}[1]{\textcolor[rgb]{0.53,0.00,0.00}{{#1}}}
    \newcommand{\SpecialCharTok}[1]{\textcolor[rgb]{0.25,0.44,0.63}{{#1}}}
    \newcommand{\VerbatimStringTok}[1]{\textcolor[rgb]{0.25,0.44,0.63}{{#1}}}
    \newcommand{\SpecialStringTok}[1]{\textcolor[rgb]{0.73,0.40,0.53}{{#1}}}
    \newcommand{\ImportTok}[1]{{#1}}
    \newcommand{\DocumentationTok}[1]{\textcolor[rgb]{0.73,0.13,0.13}{\textit{{#1}}}}
    \newcommand{\AnnotationTok}[1]{\textcolor[rgb]{0.38,0.63,0.69}{\textbf{\textit{{#1}}}}}
    \newcommand{\CommentVarTok}[1]{\textcolor[rgb]{0.38,0.63,0.69}{\textbf{\textit{{#1}}}}}
    \newcommand{\VariableTok}[1]{\textcolor[rgb]{0.10,0.09,0.49}{{#1}}}
    \newcommand{\ControlFlowTok}[1]{\textcolor[rgb]{0.00,0.44,0.13}{\textbf{{#1}}}}
    \newcommand{\OperatorTok}[1]{\textcolor[rgb]{0.40,0.40,0.40}{{#1}}}
    \newcommand{\BuiltInTok}[1]{{#1}}
    \newcommand{\ExtensionTok}[1]{{#1}}
    \newcommand{\PreprocessorTok}[1]{\textcolor[rgb]{0.74,0.48,0.00}{{#1}}}
    \newcommand{\AttributeTok}[1]{\textcolor[rgb]{0.49,0.56,0.16}{{#1}}}
    \newcommand{\InformationTok}[1]{\textcolor[rgb]{0.38,0.63,0.69}{\textbf{\textit{{#1}}}}}
    \newcommand{\WarningTok}[1]{\textcolor[rgb]{0.38,0.63,0.69}{\textbf{\textit{{#1}}}}}


    % Define a nice break command that doesn't care if a line doesn't already
    % exist.
    \def\br{\hspace*{\fill} \\* }
    % Math Jax compatibility definitions
    \def\gt{>}
    \def\lt{<}
    \let\Oldtex\TeX
    \let\Oldlatex\LaTeX
    \renewcommand{\TeX}{\textrm{\Oldtex}}
    \renewcommand{\LaTeX}{\textrm{\Oldlatex}}
    % Document parameters
    % Document title
    \title{Project 2}
    \vspace{5px}
    \author{author: Alexander Krasovskiy}
    
    
    
    
% Pygments definitions
\makeatletter
\def\PY@reset{\let\PY@it=\relax \let\PY@bf=\relax%
    \let\PY@ul=\relax \let\PY@tc=\relax%
    \let\PY@bc=\relax \let\PY@ff=\relax}
\def\PY@tok#1{\csname PY@tok@#1\endcsname}
\def\PY@toks#1+{\ifx\relax#1\empty\else%
    \PY@tok{#1}\expandafter\PY@toks\fi}
\def\PY@do#1{\PY@bc{\PY@tc{\PY@ul{%
    \PY@it{\PY@bf{\PY@ff{#1}}}}}}}
\def\PY#1#2{\PY@reset\PY@toks#1+\relax+\PY@do{#2}}

\@namedef{PY@tok@w}{\def\PY@tc##1{\textcolor[rgb]{0.73,0.73,0.73}{##1}}}
\@namedef{PY@tok@c}{\let\PY@it=\textit\def\PY@tc##1{\textcolor[rgb]{0.24,0.48,0.48}{##1}}}
\@namedef{PY@tok@cp}{\def\PY@tc##1{\textcolor[rgb]{0.61,0.40,0.00}{##1}}}
\@namedef{PY@tok@k}{\let\PY@bf=\textbf\def\PY@tc##1{\textcolor[rgb]{0.00,0.50,0.00}{##1}}}
\@namedef{PY@tok@kp}{\def\PY@tc##1{\textcolor[rgb]{0.00,0.50,0.00}{##1}}}
\@namedef{PY@tok@kt}{\def\PY@tc##1{\textcolor[rgb]{0.69,0.00,0.25}{##1}}}
\@namedef{PY@tok@o}{\def\PY@tc##1{\textcolor[rgb]{0.40,0.40,0.40}{##1}}}
\@namedef{PY@tok@ow}{\let\PY@bf=\textbf\def\PY@tc##1{\textcolor[rgb]{0.67,0.13,1.00}{##1}}}
\@namedef{PY@tok@nb}{\def\PY@tc##1{\textcolor[rgb]{0.00,0.50,0.00}{##1}}}
\@namedef{PY@tok@nf}{\def\PY@tc##1{\textcolor[rgb]{0.00,0.00,1.00}{##1}}}
\@namedef{PY@tok@nc}{\let\PY@bf=\textbf\def\PY@tc##1{\textcolor[rgb]{0.00,0.00,1.00}{##1}}}
\@namedef{PY@tok@nn}{\let\PY@bf=\textbf\def\PY@tc##1{\textcolor[rgb]{0.00,0.00,1.00}{##1}}}
\@namedef{PY@tok@ne}{\let\PY@bf=\textbf\def\PY@tc##1{\textcolor[rgb]{0.80,0.25,0.22}{##1}}}
\@namedef{PY@tok@nv}{\def\PY@tc##1{\textcolor[rgb]{0.10,0.09,0.49}{##1}}}
\@namedef{PY@tok@no}{\def\PY@tc##1{\textcolor[rgb]{0.53,0.00,0.00}{##1}}}
\@namedef{PY@tok@nl}{\def\PY@tc##1{\textcolor[rgb]{0.46,0.46,0.00}{##1}}}
\@namedef{PY@tok@ni}{\let\PY@bf=\textbf\def\PY@tc##1{\textcolor[rgb]{0.44,0.44,0.44}{##1}}}
\@namedef{PY@tok@na}{\def\PY@tc##1{\textcolor[rgb]{0.41,0.47,0.13}{##1}}}
\@namedef{PY@tok@nt}{\let\PY@bf=\textbf\def\PY@tc##1{\textcolor[rgb]{0.00,0.50,0.00}{##1}}}
\@namedef{PY@tok@nd}{\def\PY@tc##1{\textcolor[rgb]{0.67,0.13,1.00}{##1}}}
\@namedef{PY@tok@s}{\def\PY@tc##1{\textcolor[rgb]{0.73,0.13,0.13}{##1}}}
\@namedef{PY@tok@sd}{\let\PY@it=\textit\def\PY@tc##1{\textcolor[rgb]{0.73,0.13,0.13}{##1}}}
\@namedef{PY@tok@si}{\let\PY@bf=\textbf\def\PY@tc##1{\textcolor[rgb]{0.64,0.35,0.47}{##1}}}
\@namedef{PY@tok@se}{\let\PY@bf=\textbf\def\PY@tc##1{\textcolor[rgb]{0.67,0.36,0.12}{##1}}}
\@namedef{PY@tok@sr}{\def\PY@tc##1{\textcolor[rgb]{0.64,0.35,0.47}{##1}}}
\@namedef{PY@tok@ss}{\def\PY@tc##1{\textcolor[rgb]{0.10,0.09,0.49}{##1}}}
\@namedef{PY@tok@sx}{\def\PY@tc##1{\textcolor[rgb]{0.00,0.50,0.00}{##1}}}
\@namedef{PY@tok@m}{\def\PY@tc##1{\textcolor[rgb]{0.40,0.40,0.40}{##1}}}
\@namedef{PY@tok@gh}{\let\PY@bf=\textbf\def\PY@tc##1{\textcolor[rgb]{0.00,0.00,0.50}{##1}}}
\@namedef{PY@tok@gu}{\let\PY@bf=\textbf\def\PY@tc##1{\textcolor[rgb]{0.50,0.00,0.50}{##1}}}
\@namedef{PY@tok@gd}{\def\PY@tc##1{\textcolor[rgb]{0.63,0.00,0.00}{##1}}}
\@namedef{PY@tok@gi}{\def\PY@tc##1{\textcolor[rgb]{0.00,0.52,0.00}{##1}}}
\@namedef{PY@tok@gr}{\def\PY@tc##1{\textcolor[rgb]{0.89,0.00,0.00}{##1}}}
\@namedef{PY@tok@ge}{\let\PY@it=\textit}
\@namedef{PY@tok@gs}{\let\PY@bf=\textbf}
\@namedef{PY@tok@gp}{\let\PY@bf=\textbf\def\PY@tc##1{\textcolor[rgb]{0.00,0.00,0.50}{##1}}}
\@namedef{PY@tok@go}{\def\PY@tc##1{\textcolor[rgb]{0.44,0.44,0.44}{##1}}}
\@namedef{PY@tok@gt}{\def\PY@tc##1{\textcolor[rgb]{0.00,0.27,0.87}{##1}}}
\@namedef{PY@tok@err}{\def\PY@bc##1{{\setlength{\fboxsep}{\string -\fboxrule}\fcolorbox[rgb]{1.00,0.00,0.00}{1,1,1}{\strut ##1}}}}
\@namedef{PY@tok@kc}{\let\PY@bf=\textbf\def\PY@tc##1{\textcolor[rgb]{0.00,0.50,0.00}{##1}}}
\@namedef{PY@tok@kd}{\let\PY@bf=\textbf\def\PY@tc##1{\textcolor[rgb]{0.00,0.50,0.00}{##1}}}
\@namedef{PY@tok@kn}{\let\PY@bf=\textbf\def\PY@tc##1{\textcolor[rgb]{0.00,0.50,0.00}{##1}}}
\@namedef{PY@tok@kr}{\let\PY@bf=\textbf\def\PY@tc##1{\textcolor[rgb]{0.00,0.50,0.00}{##1}}}
\@namedef{PY@tok@bp}{\def\PY@tc##1{\textcolor[rgb]{0.00,0.50,0.00}{##1}}}
\@namedef{PY@tok@fm}{\def\PY@tc##1{\textcolor[rgb]{0.00,0.00,1.00}{##1}}}
\@namedef{PY@tok@vc}{\def\PY@tc##1{\textcolor[rgb]{0.10,0.09,0.49}{##1}}}
\@namedef{PY@tok@vg}{\def\PY@tc##1{\textcolor[rgb]{0.10,0.09,0.49}{##1}}}
\@namedef{PY@tok@vi}{\def\PY@tc##1{\textcolor[rgb]{0.10,0.09,0.49}{##1}}}
\@namedef{PY@tok@vm}{\def\PY@tc##1{\textcolor[rgb]{0.10,0.09,0.49}{##1}}}
\@namedef{PY@tok@sa}{\def\PY@tc##1{\textcolor[rgb]{0.73,0.13,0.13}{##1}}}
\@namedef{PY@tok@sb}{\def\PY@tc##1{\textcolor[rgb]{0.73,0.13,0.13}{##1}}}
\@namedef{PY@tok@sc}{\def\PY@tc##1{\textcolor[rgb]{0.73,0.13,0.13}{##1}}}
\@namedef{PY@tok@dl}{\def\PY@tc##1{\textcolor[rgb]{0.73,0.13,0.13}{##1}}}
\@namedef{PY@tok@s2}{\def\PY@tc##1{\textcolor[rgb]{0.73,0.13,0.13}{##1}}}
\@namedef{PY@tok@sh}{\def\PY@tc##1{\textcolor[rgb]{0.73,0.13,0.13}{##1}}}
\@namedef{PY@tok@s1}{\def\PY@tc##1{\textcolor[rgb]{0.73,0.13,0.13}{##1}}}
\@namedef{PY@tok@mb}{\def\PY@tc##1{\textcolor[rgb]{0.40,0.40,0.40}{##1}}}
\@namedef{PY@tok@mf}{\def\PY@tc##1{\textcolor[rgb]{0.40,0.40,0.40}{##1}}}
\@namedef{PY@tok@mh}{\def\PY@tc##1{\textcolor[rgb]{0.40,0.40,0.40}{##1}}}
\@namedef{PY@tok@mi}{\def\PY@tc##1{\textcolor[rgb]{0.40,0.40,0.40}{##1}}}
\@namedef{PY@tok@il}{\def\PY@tc##1{\textcolor[rgb]{0.40,0.40,0.40}{##1}}}
\@namedef{PY@tok@mo}{\def\PY@tc##1{\textcolor[rgb]{0.40,0.40,0.40}{##1}}}
\@namedef{PY@tok@ch}{\let\PY@it=\textit\def\PY@tc##1{\textcolor[rgb]{0.24,0.48,0.48}{##1}}}
\@namedef{PY@tok@cm}{\let\PY@it=\textit\def\PY@tc##1{\textcolor[rgb]{0.24,0.48,0.48}{##1}}}
\@namedef{PY@tok@cpf}{\let\PY@it=\textit\def\PY@tc##1{\textcolor[rgb]{0.24,0.48,0.48}{##1}}}
\@namedef{PY@tok@c1}{\let\PY@it=\textit\def\PY@tc##1{\textcolor[rgb]{0.24,0.48,0.48}{##1}}}
\@namedef{PY@tok@cs}{\let\PY@it=\textit\def\PY@tc##1{\textcolor[rgb]{0.24,0.48,0.48}{##1}}}

\def\PYZbs{\char`\\}
\def\PYZus{\char`\_}
\def\PYZob{\char`\{}
\def\PYZcb{\char`\}}
\def\PYZca{\char`\^}
\def\PYZam{\char`\&}
\def\PYZlt{\char`\<}
\def\PYZgt{\char`\>}
\def\PYZsh{\char`\#}
\def\PYZpc{\char`\%}
\def\PYZdl{\char`\$}
\def\PYZhy{\char`\-}
\def\PYZsq{\char`\'}
\def\PYZdq{\char`\"}
\def\PYZti{\char`\~}
% for compatibility with earlier versions
\def\PYZat{@}
\def\PYZlb{[}
\def\PYZrb{]}
\makeatother


    % For linebreaks inside Verbatim environment from package fancyvrb.
    \makeatletter
        \newbox\Wrappedcontinuationbox
        \newbox\Wrappedvisiblespacebox
        \newcommand*\Wrappedvisiblespace {\textcolor{red}{\textvisiblespace}}
        \newcommand*\Wrappedcontinuationsymbol {\textcolor{red}{\llap{\tiny$\m@th\hookrightarrow$}}}
        \newcommand*\Wrappedcontinuationindent {3ex }
        \newcommand*\Wrappedafterbreak {\kern\Wrappedcontinuationindent\copy\Wrappedcontinuationbox}
        % Take advantage of the already applied Pygments mark-up to insert
        % potential linebreaks for TeX processing.
        %        {, <, #, %, $, ' and ": go to next line.
        %        _, }, ^, &, >, - and ~: stay at end of broken line.
        % Use of \textquotesingle for straight quote.
        \newcommand*\Wrappedbreaksatspecials {%
            \def\PYGZus{\discretionary{\char`\_}{\Wrappedafterbreak}{\char`\_}}%
            \def\PYGZob{\discretionary{}{\Wrappedafterbreak\char`\{}{\char`\{}}%
            \def\PYGZcb{\discretionary{\char`\}}{\Wrappedafterbreak}{\char`\}}}%
            \def\PYGZca{\discretionary{\char`\^}{\Wrappedafterbreak}{\char`\^}}%
            \def\PYGZam{\discretionary{\char`\&}{\Wrappedafterbreak}{\char`\&}}%
            \def\PYGZlt{\discretionary{}{\Wrappedafterbreak\char`\<}{\char`\<}}%
            \def\PYGZgt{\discretionary{\char`\>}{\Wrappedafterbreak}{\char`\>}}%
            \def\PYGZsh{\discretionary{}{\Wrappedafterbreak\char`\#}{\char`\#}}%
            \def\PYGZpc{\discretionary{}{\Wrappedafterbreak\char`\%}{\char`\%}}%
            \def\PYGZdl{\discretionary{}{\Wrappedafterbreak\char`\$}{\char`\$}}%
            \def\PYGZhy{\discretionary{\char`\-}{\Wrappedafterbreak}{\char`\-}}%
            \def\PYGZsq{\discretionary{}{\Wrappedafterbreak\textquotesingle}{\textquotesingle}}%
            \def\PYGZdq{\discretionary{}{\Wrappedafterbreak\char`\"}{\char`\"}}%
            \def\PYGZti{\discretionary{\char`\~}{\Wrappedafterbreak}{\char`\~}}%
        }
        % Some characters . , ; ? ! / are not pygmentized.
        % This macro makes them "active" and they will insert potential linebreaks
        \newcommand*\Wrappedbreaksatpunct {%
            \lccode`\~`\.\lowercase{\def~}{\discretionary{\hbox{\char`\.}}{\Wrappedafterbreak}{\hbox{\char`\.}}}%
            \lccode`\~`\,\lowercase{\def~}{\discretionary{\hbox{\char`\,}}{\Wrappedafterbreak}{\hbox{\char`\,}}}%
            \lccode`\~`\;\lowercase{\def~}{\discretionary{\hbox{\char`\;}}{\Wrappedafterbreak}{\hbox{\char`\;}}}%
            \lccode`\~`\:\lowercase{\def~}{\discretionary{\hbox{\char`\:}}{\Wrappedafterbreak}{\hbox{\char`\:}}}%
            \lccode`\~`\?\lowercase{\def~}{\discretionary{\hbox{\char`\?}}{\Wrappedafterbreak}{\hbox{\char`\?}}}%
            \lccode`\~`\!\lowercase{\def~}{\discretionary{\hbox{\char`\!}}{\Wrappedafterbreak}{\hbox{\char`\!}}}%
            \lccode`\~`\/\lowercase{\def~}{\discretionary{\hbox{\char`\/}}{\Wrappedafterbreak}{\hbox{\char`\/}}}%
            \catcode`\.\active
            \catcode`\,\active
            \catcode`\;\active
            \catcode`\:\active
            \catcode`\?\active
            \catcode`\!\active
            \catcode`\/\active
            \lccode`\~`\~
        }
    \makeatother

    \let\OriginalVerbatim=\Verbatim
    \makeatletter
    \renewcommand{\Verbatim}[1][1]{%
        %\parskip\z@skip
        \sbox\Wrappedcontinuationbox {\Wrappedcontinuationsymbol}%
        \sbox\Wrappedvisiblespacebox {\FV@SetupFont\Wrappedvisiblespace}%
        \def\FancyVerbFormatLine ##1{\hsize\linewidth
            \vtop{\raggedright\hyphenpenalty\z@\exhyphenpenalty\z@
                \doublehyphendemerits\z@\finalhyphendemerits\z@
                \strut ##1\strut}%
        }%
        % If the linebreak is at a space, the latter will be displayed as visible
        % space at end of first line, and a continuation symbol starts next line.
        % Stretch/shrink are however usually zero for typewriter font.
        \def\FV@Space {%
            \nobreak\hskip\z@ plus\fontdimen3\font minus\fontdimen4\font
            \discretionary{\copy\Wrappedvisiblespacebox}{\Wrappedafterbreak}
            {\kern\fontdimen2\font}%
        }%

        % Allow breaks at special characters using \PYG... macros.
        \Wrappedbreaksatspecials
        % Breaks at punctuation characters . , ; ? ! and / need catcode=\active
        \OriginalVerbatim[#1,codes*=\Wrappedbreaksatpunct]%
    }
    \makeatother

    % Exact colors from NB
    \definecolor{incolor}{HTML}{303F9F}
    \definecolor{outcolor}{HTML}{D84315}
    \definecolor{cellborder}{HTML}{CFCFCF}
    \definecolor{cellbackground}{HTML}{F7F7F7}

    % prompt
    \makeatletter
    \newcommand{\boxspacing}{\kern\kvtcb@left@rule\kern\kvtcb@boxsep}
    \makeatother
    \newcommand{\prompt}[4]{
        {\ttfamily\llap{{\color{#2}[#3]:\hspace{3pt}#4}}\vspace{-\baselineskip}}
    }
    

    
    % Prevent overflowing lines due to hard-to-break entities
    \sloppy
    % Setup hyperref package
    \hypersetup{
      breaklinks=true,  % so long urls are correctly broken across lines
      colorlinks=true,
      urlcolor=urlcolor,
      linkcolor=linkcolor,
      citecolor=citecolor,
      }
    % Slightly bigger margins than the latex defaults
    
    \geometry{verbose,tmargin=1in,bmargin=1in,lmargin=1in,rmargin=1in}
    
    

\begin{document}
    
    \maketitle
    
    

    \hypertarget{task-2.1-documentation}{%
\subsubsection{Task 2.1 Documentation}\label{task-2.1-documentation}}

    \hypertarget{data-source}{%
\paragraph{1. Data Source:}\label{data-source}}

\begin{itemize}
\tightlist
\item
  The data source for the Task 2.1 is a generated random uniformly
  distributed sample of size n (scipy.stats.uniform module is used)
\item
  Initial assumptions: X is a uniformly distributed sample with values in
  range \([0,\theta]\) (where \(\theta\) is a model parameter)
\end{itemize}

    \hypertarget{research-purpose}{%
\subparagraph{2. Research: }\label{research-purpose}}

\begin{enumerate}
\def\labelenumi{\arabic{enumi}.}
\tightlist
\item
  Test data (generated):
\end{enumerate}

\begin{itemize}
\tightlist
\item
  \(X \in \text{\{ X1, X2 , ..., X100 \}}\)
\item
  Size of the generated data:
  \(\small{\text{|X1| = 1000, |X2| = 2000, ... , |X100| = 100000}}\)
\end{itemize}

\vspace{15px}
\begin{enumerate}
\def\labelenumi{\arabic{enumi}.}
\setcounter{enumi}{1}
\tightlist
\item
  Possible \(\theta\) estimators:
\end{enumerate}

\begin{itemize}
\tightlist
\item
  Min(X) = \(\space\space X_{1:n}\space\space\)  -  Sample X minimum
\item
  Mean(X) =
  \(\space\space (X_{1:n}, X_{n:n})\space\space ---||--- \text{mean}\)
\item
  Max(X) = \(\space\space X_{n:n}\space\space ---||--- \text{maximum}\)
\end{itemize}

\vspace{15px}
\begin{enumerate}
\def\labelenumi{\arabic{enumi}.}
\setcounter{enumi}{2}
\tightlist
\item
  Parameter \(\theta = 10\) (or any natural number)
  \vspace{15px}
\item
  Expected test result:
  \(\textbf{Max(X)}\text{ is a sufficient statistics for the parameter }\theta\)
  \vspace{15px}
  \item
  Alternative result:
  \(\textbf{Mean(X)} \text{ or }\textbf{Min(X)}\text{ is a sufficient statistics for the parameter }\theta\)
\end{enumerate}
\vspace{15px}
\begin{itemize}
  \tightlist
  \item
    \(\text{Research purpose: check if the expected test result is correct}\newline\)
    \(\text{(Show that }\textbf{ Max(x) }\text{is a sufficient estimator)}\)
  \end{itemize}

    \vspace{120px}

    \begin{tcolorbox}[breakable, size=fbox, boxrule=1pt, pad at break*=1mm,colback=cellbackground, colframe=cellborder]
\prompt{In}{incolor}{48}{\boxspacing}
\begin{Verbatim}[commandchars=\\\{\}]
\PY{n}{theta} \PY{o}{=} \PY{l+m+mi}{10}
\PY{c+c1}{\PYZsh{} Define sample (X) generator}
\PY{k}{def} \PY{n+nf}{T}\PY{p}{(}\PY{n}{theta}\PY{p}{,} \PY{n}{sample\PYZus{}size}\PY{p}{)}\PY{p}{:}
    \PY{k}{return} \PY{n}{uniform}\PY{o}{.}\PY{n}{rvs}\PY{p}{(}\PY{l+m+mi}{0}\PY{p}{,}\PY{n}{theta}\PY{p}{,}\PY{n}{sample\PYZus{}size}\PY{p}{)}
\end{Verbatim}
\end{tcolorbox}

    \begin{tcolorbox}[breakable, size=fbox, boxrule=1pt, pad at break*=1mm,colback=cellbackground, colframe=cellborder]
\prompt{In}{incolor}{49}{\boxspacing}
\begin{Verbatim}[commandchars=\\\{\}]
\PY{c+c1}{\PYZsh{}Define statistics functions}
\PY{k}{def} \PY{n+nf}{Mean}\PY{p}{(}\PY{n}{sample}\PY{p}{)}\PY{p}{:}
    \PY{k}{return} \PY{n}{sample}\PY{o}{.}\PY{n}{mean}\PY{p}{(}\PY{p}{)}

\PY{k}{def} \PY{n+nf}{Min}\PY{p}{(}\PY{n}{sample}\PY{p}{)}\PY{p}{:}
    \PY{k}{return} \PY{n}{sample}\PY{o}{.}\PY{n}{min}\PY{p}{(}\PY{p}{)}

\PY{k}{def} \PY{n+nf}{Max}\PY{p}{(}\PY{n}{sample}\PY{p}{)}\PY{p}{:}
    \PY{k}{return} \PY{n}{sample}\PY{o}{.}\PY{n}{max}\PY{p}{(}\PY{p}{)}
\end{Verbatim}
\end{tcolorbox}

    

    \hypertarget{introduction-probability-distribution-analysis}{%
\subparagraph{4. Probability distribution
analysis}\label{introduction-probability-distribution-analysis}}

\begin{enumerate}
\def\labelenumi{\arabic{enumi}.}
\tightlist
\item
  The \(\small{\textbf{Kolmogorov-Smirnov}}\) is a nonparametric test of the equality of continuous (or discontinuous), one-dimensional probability distributions that can be used to compare a sample with a reference probability distribution (one-sample K–S test), or to compare two samples (two-sample K–S test). In essence, the test answers the question "What is the probability that this collection of samples could have been drawn from that probability distribution?"
\item
  \(\text{The solution depends on a function scipy.stats.kstest that implements the}\small{\textbf{ Kolmogorov-Smirnov }}\text{test.}\)
\item
  A brief description of the Kolmogorov-Smirnov test:
\end{enumerate}

\begin{itemize}
\tightlist
\item
  method: scipy.stats.kstest(rvs, cdf, args=(), N=20,
  alternative=`two-sided', method=`auto')
\item
  reference:
  https://docs.scipy.org/doc/scipy/reference/generated/scipy.stats.kstest.html
\item
  Performs the (one-sample or two-sample) Kolmogorov-Smirnov test for
  goodness of fit. The one-sample test compares the underlying
  distribution F(x) of a sample against a given distribution G(x)
  \(\textbf{(this option is used)}\). The two-sample test compares the
  underlying distributions of two independent samples. Both tests are
  valid only for continuous distributions.
\item
  parameters:

  \begin{itemize}
  \tightlist
  \item
    rvs - random values \(\equiv X\)
  \item
    cdf - expected cumulative distribution function
    \(\equiv \text{Unif}(0,\theta)\)
  \item
    N - sample size \(\equiv \text{size of } X\)
  \item
    If the sample comes from distribution \(F(x) \equiv cdf\), then
    \(D_n\) converges to 0 almost surely in the limit when
    \({\displaystyle n}\) goes to infinity.
  \end{itemize}
  \item
  Let \(D_{n}=\sup _{x}|F_{n}(x)-F(x)|\)
  \item
  H0 (K-S Test): \({\sqrt {n}}D_{n} \space\space\) converges to the Kolmogorov
  distribution, which does not depend on the empirical CDF \(F(x)\)

\end{itemize}

\vspace{80px}
    \begin{tcolorbox}[breakable, size=fbox, boxrule=1pt, pad at break*=1mm,colback=cellbackground, colframe=cellborder]
\prompt{In}{incolor}{50}{\boxspacing}
\begin{Verbatim}[commandchars=\\\{\}]
\PY{k+kn}{from} \PY{n+nn}{scipy}\PY{n+nn}{.}\PY{n+nn}{stats} \PY{k+kn}{import} \PY{n}{kstest}
\PY{n}{Lambda} \PY{o}{=} \PY{l+m+mf}{0.05}
\PY{n}{n} \PY{o}{=} \PY{l+m+mi}{100000}
\PY{n}{test} \PY{o}{=} \PY{n}{kstest}\PY{p}{(}\PY{n}{T}\PY{p}{(}\PY{n}{theta}\PY{p}{,}\PY{n}{n}\PY{p}{)}\PY{p}{,}\PY{n}{uniform}\PY{o}{.}\PY{n}{cdf}\PY{p}{,} \PY{n}{N}\PY{o}{=}\PY{n}{n}\PY{p}{,} \PY{n}{args}\PY{o}{=}\PY{p}{(}\PY{l+m+mi}{0}\PY{p}{,}\PY{n}{theta}\PY{p}{)}\PY{p}{)}
\PY{n}{s}\PY{p}{,} \PY{n}{p\PYZus{}value} \PY{o}{=} \PY{n}{test}
\PY{n+nb}{print}\PY{p}{(}\PY{l+s+s2}{\PYZdq{}}\PY{l+s+s2}{sample size = }\PY{l+s+si}{\PYZpc{}d}\PY{l+s+s2}{\PYZdq{}}\PY{o}{\PYZpc{}}\PY{k}{n})
\PY{n+nb}{print}\PY{p}{(}\PY{n}{test}\PY{p}{)}
\PY{k}{if} \PY{p}{(}\PY{n}{p\PYZus{}value}\PY{o}{\PYZgt{}}\PY{n}{Lambda}\PY{p}{)}\PY{p}{:}
    \PY{n+nb}{print}\PY{p}{(}\PY{p}{)}
    \PY{n+nb}{print}\PY{p}{(}\PY{l+s+s2}{\PYZdq{}}\PY{l+s+s2}{H0 for the Kolmogorov\PYZhy{}Smirnov test is correct, lambda = }\PY{l+s+si}{\PYZpc{}.3f}\PY{l+s+s2}{ (}\PY{l+s+si}{\PYZpc{}d}\PY{l+s+si}{\PYZpc{}\PYZpc{}}\PY{l+s+s2}{ percentile)}\PY{l+s+s2}{\PYZdq{}}\PY{o}{\PYZpc{}}\PY{p}{(}\PY{n}{Lambda}\PY{p}{,}\PY{l+m+mi}{100}\PY{o}{\PYZhy{}}\PY{l+m+mi}{100}\PY{o}{*}\PY{n}{Lambda}\PY{p}{)}\PY{p}{)}
    \PY{n+nb}{print}\PY{p}{(}\PY{p}{)}
    \PY{n+nb}{print}\PY{p}{(}\PY{l+s+s2}{\PYZdq{}}\PY{l+s+s2}{The sample X distribution is likely to be equivalent to Unif[0,theta]}\PY{l+s+s2}{\PYZdq{}}\PY{p}{)}
\PY{k}{else}\PY{p}{:}
    \PY{n+nb}{print}\PY{p}{(}\PY{l+s+s2}{\PYZdq{}}\PY{l+s+s2}{H0 for the Kolmogorov\PYZhy{}Smirnov test is not correct, lambda = }\PY{l+s+si}{\PYZpc{}.3f}\PY{l+s+s2}{\PYZdq{}}\PY{o}{\PYZpc{}}\PY{k}{Lambda})
    \PY{n+nb}{print}\PY{p}{(}\PY{l+s+s2}{\PYZdq{}}\PY{l+s+s2}{X distribution is unlikely to be equivalent to Unif[0,theta]}\PY{l+s+s2}{\PYZdq{}}\PY{p}{)}
\end{Verbatim}
\end{tcolorbox}

    \begin{Verbatim}[commandchars=\\\{\}]
sample size = 100000
KstestResult(statistic=0.0026661944797430337, pvalue=0.4750146065722749)

The Kolmogorov-Smirnov test H0 is correct, lambda = 0.050 (95 percentile)

The sample X distribution is equivalent to Unif[0,theta] with probability >= 95\%
    \end{Verbatim}

    \begin{enumerate}
\def\labelenumi{\arabic{enumi}.}
\setcounter{enumi}{3}
\tightlist
\item
As expected, the p-value of 0.47 is not below our threshold of 0.05, so we cannot reject the null hypothesis.  
=>\end{enumerate}

\begin{itemize}
\tightlist
\item
  \text{X is a random sample with the distribution}
  = \text{Unif}(0,\theta) \text{ with a probability >= 95\%}
\end{itemize}

    \begin{enumerate}
\def\labelenumi{\arabic{enumi}.}
\setcounter{enumi}{4}
\tightlist
\item
  Additionally, it is necessary to check, that the sample generated by T
  is simple.
\end{enumerate}
\begin{itemize}
\tightlist
\item
  function isSimpleRandomSample(T, theta,
  expected\_range, eps = \(10^{-5}\)) - to check if a uniformly distributed
  random variable T is a simple random sample for the expected range
  {[}a,b{]}.

  \begin{enumerate}
  \def\labelenumi{\arabic{enumi}.}
  \tightlist
  \item
    Calculate the expected mean and standard deviation of the
    Uniform{[}a,b{]} distribution
  \item
    Iterate for each sample size =n min: \(10\), max: \(10^7\)
  \item

  For each sample size (=n):   
  \begin{itemize}
    \tightlist
    \item
    Generate a random sample of size n 
    \item
    Calculate {[}a\_iter,b\_iter{]} == {[}min(sample),max(sample){]} 
    \item
    Compute p == P(x\(\lt\)a\_iter or x\(\gt\)b\_iter) 
    \item
    Calculate the sample mean and standard deviation 
    \item
    In order to prove that T(n) is a simple random sample, we need to check that the random variable T(n) is not limited for the population {[}0,theta{]} 
    \item
    Therefore, it is necessary and sufficient to check that lim(p)=0, n-\textgreater inf for a given eps. 
    \item
    =\textgreater{} If abs(p) \textless{} eps: T generates a simple random sample.
    
    \begin{itemize}
    \tightlist
    \item
      return True
    \end{itemize}
    
  \end{itemize}
  \item
    Else: - return False
  \end{enumerate}
\end{itemize}

    \begin{tcolorbox}[breakable, size=fbox, boxrule=1pt, pad at break*=1mm,colback=cellbackground, colframe=cellborder]
\prompt{In}{incolor}{57}{\boxspacing}
\begin{Verbatim}[commandchars=\\\{\}]
\PY{k+kn}{from} \PY{n+nn}{math} \PY{k+kn}{import} \PY{n}{sqrt}

\PY{k}{def} \PY{n+nf}{isSimpleRandomSample}\PY{p}{(}\PY{n}{T}\PY{p}{,} \PY{n}{theta}\PY{p}{,} \PY{n}{expected\PYZus{}range}\PY{p}{,} \PY{n}{eps}\PY{p}{)}\PY{p}{:}
    \PY{n}{a\PYZus{}iter}\PY{o}{=}\PY{k+kc}{None}
    \PY{n}{b\PYZus{}iter}\PY{o}{=}\PY{k+kc}{None}
    \PY{n}{diff} \PY{o}{=} \PY{n+nb}{abs}\PY{p}{(}\PY{n}{expected\PYZus{}range}\PY{p}{[}\PY{l+m+mi}{0}\PY{p}{]}\PY{o}{\PYZhy{}}\PY{n}{expected\PYZus{}range}\PY{p}{[}\PY{l+m+mi}{1}\PY{p}{]}\PY{p}{)}
    \PY{n}{sample\PYZus{}sizes} \PY{o}{=} \PY{p}{[}\PY{n+nb}{int}\PY{p}{(}\PY{n}{x}\PY{p}{)} \PY{k}{for} \PY{n}{x} \PY{o+ow}{in} \PY{p}{[}\PY{l+m+mi}{10}\PY{p}{,}\PY{l+m+mf}{1e2}\PY{p}{,}\PY{l+m+mf}{1e3}\PY{p}{,}\PY{l+m+mf}{1e4}\PY{p}{,}\PY{l+m+mf}{1e5}\PY{p}{,}\PY{l+m+mf}{1e6}\PY{p}{,} \PY{l+m+mf}{1e7}\PY{p}{]}\PY{p}{]}
    \PY{n+nb}{print}\PY{p}{(}\PY{l+s+s2}{\PYZdq{}}\PY{l+s+s2}{expected range:}\PY{l+s+s2}{\PYZdq{}}\PY{p}{,}\PY{n}{expected\PYZus{}range}\PY{p}{)}
    \PY{n}{expected\PYZus{}mean} \PY{o}{=} \PY{p}{(}\PY{n}{expected\PYZus{}range}\PY{p}{[}\PY{l+m+mi}{1}\PY{p}{]}\PY{o}{\PYZhy{}}\PY{n}{expected\PYZus{}range}\PY{p}{[}\PY{l+m+mi}{0}\PY{p}{]}\PY{p}{)}\PY{o}{/}\PY{l+m+mi}{2}
    \PY{n}{expected\PYZus{}std} \PY{o}{=} \PY{n}{sqrt}\PY{p}{(}\PY{l+m+mi}{1}\PY{o}{/}\PY{l+m+mi}{12} \PY{o}{*} \PY{p}{(}\PY{n}{expected\PYZus{}range}\PY{p}{[}\PY{l+m+mi}{1}\PY{p}{]}\PY{o}{\PYZhy{}}\PY{n}{expected\PYZus{}range}\PY{p}{[}\PY{l+m+mi}{0}\PY{p}{]}\PY{p}{)}\PY{o}{*}\PY{o}{*}\PY{l+m+mi}{2}\PY{p}{)}
    \PY{n+nb}{print}\PY{p}{(}\PY{l+s+s2}{\PYZdq{}}\PY{l+s+s2}{expected mean:}\PY{l+s+s2}{\PYZdq{}}\PY{p}{,}\PY{n}{expected\PYZus{}mean}\PY{p}{)}
    \PY{n+nb}{print}\PY{p}{(}\PY{l+s+s2}{\PYZdq{}}\PY{l+s+s2}{expected standard deviation:}\PY{l+s+s2}{\PYZdq{}}\PY{p}{,}\PY{n}{expected\PYZus{}std}\PY{p}{)}
    \PY{n+nb}{print}\PY{p}{(}\PY{p}{)}
    \PY{k}{for} \PY{n}{size} \PY{o+ow}{in} \PY{n}{sample\PYZus{}sizes}\PY{p}{:}
        \PY{n}{samp} \PY{o}{=} \PY{n}{T}\PY{p}{(}\PY{n}{theta}\PY{p}{,} \PY{n}{size}\PY{p}{)}
        \PY{n}{a\PYZus{}iter} \PY{o}{=} \PY{n}{samp}\PY{o}{.}\PY{n}{min}\PY{p}{(}\PY{p}{)}
        \PY{n}{b\PYZus{}iter} \PY{o}{=} \PY{n}{samp}\PY{o}{.}\PY{n}{max}\PY{p}{(}\PY{p}{)}
        
        \PY{n}{p} \PY{o}{=} \PY{n+nb}{abs}\PY{p}{(}\PY{n}{a\PYZus{}iter}\PY{o}{\PYZhy{}}\PY{n}{expected\PYZus{}range}\PY{p}{[}\PY{l+m+mi}{0}\PY{p}{]}\PY{p}{)}\PY{o}{/}\PY{n}{diff} \PY{o}{+} \PY{n+nb}{abs}\PY{p}{(}\PY{n}{b\PYZus{}iter}\PY{o}{\PYZhy{}}\PY{n}{expected\PYZus{}range}\PY{p}{[}\PY{l+m+mi}{1}\PY{p}{]}\PY{p}{)}\PY{o}{/}\PY{n}{diff}
        \PY{n+nb}{print}\PY{p}{(}\PY{l+s+s2}{\PYZdq{}}\PY{l+s+s2}{for sample size = }\PY{l+s+si}{\PYZpc{}d}\PY{l+s+s2}{:}\PY{l+s+s2}{\PYZdq{}}\PY{o}{\PYZpc{}}\PY{k}{size})
        \PY{n+nb}{print}\PY{p}{(}\PY{l+s+s2}{\PYZdq{}}\PY{l+s+s2}{p\PYZus{}value == P(x\PYZlt{}a\PYZus{}iter or x\PYZgt{}b\PYZus{}iter) = }\PY{l+s+si}{\PYZpc{}.8f}\PY{l+s+s2}{\PYZdq{}}\PY{o}{\PYZpc{}}\PY{k}{p})
        \PY{n+nb}{print}\PY{p}{(}\PY{l+s+s2}{\PYZdq{}}\PY{l+s+s2}{sample mean = E(X):}\PY{l+s+s2}{\PYZdq{}}\PY{p}{,}\PY{n}{samp}\PY{o}{.}\PY{n}{mean}\PY{p}{(}\PY{p}{)}\PY{p}{)}
        \PY{n+nb}{print}\PY{p}{(}\PY{l+s+s2}{\PYZdq{}}\PY{l+s+s2}{sample standard deviation = sqrt(Var(X)):}\PY{l+s+s2}{\PYZdq{}}\PY{p}{,}\PY{n}{samp}\PY{o}{.}\PY{n}{std}\PY{p}{(}\PY{p}{)}\PY{p}{)}
        \PY{n+nb}{print}\PY{p}{(}\PY{l+s+s2}{\PYZdq{}}\PY{l+s+s2}{actual range: }\PY{l+s+s2}{\PYZdq{}}\PY{p}{,}\PY{p}{[}\PY{n}{a\PYZus{}iter}\PY{p}{,}\PY{n}{b\PYZus{}iter}\PY{p}{]}\PY{p}{)}
        \PY{n+nb}{print}\PY{p}{(}\PY{p}{)}
        
        \PY{k}{if} \PY{n}{p} \PY{o}{\PYZlt{}} \PY{n}{eps}\PY{p}{:}
            \PY{n+nb}{print}\PY{p}{(}\PY{l+s+s2}{\PYZdq{}}\PY{l+s+s2}{Random sample is simple, eps=}\PY{l+s+si}{\PYZpc{}e}\PY{l+s+s2}{\PYZdq{}}\PY{o}{\PYZpc{}}\PY{k}{eps})
            \PY{k}{return} \PY{k+kc}{True}
    \PY{k}{return} \PY{k+kc}{False}
    
\PY{n}{eps}\PY{o}{=}\PY{l+m+mf}{1e\PYZhy{}5}
\PY{n}{population}\PY{o}{=}\PY{p}{[}\PY{l+m+mi}{0}\PY{p}{,}\PY{n}{theta}\PY{p}{]}

\PY{k}{if} \PY{n}{isSimpleRandomSample}\PY{p}{(}\PY{n}{T}\PY{p}{,}\PY{n}{theta}\PY{p}{,}\PY{n}{population}\PY{p}{,} \PY{n}{eps}\PY{o}{=}\PY{n}{eps}\PY{p}{)}\PY{p}{:}
    \PY{n}{success}\PY{p}{(}\PY{l+s+s2}{\PYZdq{}}\PY{l+s+se}{\PYZbs{}n}\PY{l+s+s2}{T(n) generates a simple random sample for the expected population=}\PY{l+s+si}{\PYZpc{}s}\PY{l+s+s2}{, eps=}\PY{l+s+si}{\PYZpc{}e}\PY{l+s+s2}{\PYZdq{}}\PY{o}{\PYZpc{}}\PY{p}{(}\PY{n}{population}\PY{p}{,}\PY{n}{eps}\PY{p}{)}\PY{p}{)}
\PY{k}{else}\PY{p}{:}
    \PY{n}{fail}\PY{p}{(}\PY{l+s+s2}{\PYZdq{}}\PY{l+s+se}{\PYZbs{}n}\PY{l+s+s2}{T(n) does not generate a simple random sample for the expected population=}\PY{l+s+si}{\PYZpc{}s}\PY{l+s+s2}{, eps=}\PY{l+s+si}{\PYZpc{}e}\PY{l+s+s2}{\PYZdq{}}\PY{o}{\PYZpc{}}\PY{p}{(}\PY{n}{population}\PY{p}{,}\PY{n}{eps}\PY{p}{)}\PY{p}{)}
\end{Verbatim}
\end{tcolorbox}

    \begin{Verbatim}[commandchars=\\\{\}]
expected range: [0, 10]
expected mean: 5.0
expected standard deviation: 2.8867513459481287

for sample size = 10:
p\_value == P(x<a\_iter or x>b\_iter) = 0.10368617
sample mean = E(X): 5.643999061499329
sample standard deviation = sqrt(Var(X)): 2.871822291800391
actual range:  [0.9269957357854164, 9.890134010618748]

for sample size = 100:
p\_value == P(x<a\_iter or x>b\_iter) = 0.00762663
sample mean = E(X): 4.67245060952827
sample standard deviation = sqrt(Var(X)): 3.058244597728054
actual range:  [0.028974161437055335, 9.952707822710888]

for sample size = 1000:
p\_value == P(x<a\_iter or x>b\_iter) = 0.00154257
sample mean = E(X): 4.990070772551295
sample standard deviation = sqrt(Var(X)): 2.9470834430016635
actual range:  [0.009418768815103729, 9.993993082600863]

for sample size = 10000:
p\_value == P(x<a\_iter or x>b\_iter) = 0.00020790
sample mean = E(X): 5.0136269329335414
sample standard deviation = sqrt(Var(X)): 2.9008435888900697
actual range:  [0.001960486448134846, 9.999881494744043]

for sample size = 100000:
p\_value == P(x<a\_iter or x>b\_iter) = 0.00000881
sample mean = E(X): 4.99221187269252
sample standard deviation = sqrt(Var(X)): 2.8887436061886724
actual range:  [2.3191380660314564e-05, 9.999935093523947]

Random sample is simple, eps=1.000000e-05
\textcolor{ansi-green-intense}{
T(n) generates a simple random sample for the expected population=[0, 10],
eps=1.000000e-05}
    \end{Verbatim}

    \vspace{190px}
    \hypertarget{model-definition}{%
\subparagraph{5. Model definition:}\label{model-definition}}

 \((X,\{P_{\theta}, \theta \in \Theta\})\): \newline
- X = \([ 0,\space \theta]\) \newline
- \(\theta = 10\); \(\Theta = \mathbb{R}\) \newline
- \(P_{\theta} \equiv \textit{Unif}\space(0, \theta)\) \newline


\vspace{40px}
    \hypertarget{analysis}{%
\subparagraph{6. Analysis (\(\theta\) estimation):}\label{analysis}}

\begin{enumerate}
\def\labelenumi{\arabic{enumi}.}
\tightlist
\item
  Compute the mean squared error of
  \(|\theta - S(X)|\)
\end{enumerate}

\begin{itemize}
\tightlist
\item
  Where \(S(X)\) is a statistics function
\item
  The squared error is computer for every sample
  \(X \in \text{\{ X1, X2 , ..., X100 \}}\): \(E_i = |\theta - T(X_i)|\)
\item
  =\textgreater{} The mean squared error is equal to
  \(\space\frac{1}{n}(E_1 + E_2 + ... + E_{100}) = \text{MSE}\)
\end{itemize}

    Brief description of the function \textbf{error(theta, stat, T)}: \newline

    1. Arg: \(\theta\) - estimated parameter \newline

    2. Arg: \(T == T(\theta,n)\) - random sample generator \newline

    3. Arg: \(\textit{stat}\) - a reference to the statistics function of a sample T(\(\theta\), \text{sample size}) \newline
    
    4. for each sample size in range(1000,100000, with step 1000): \newline
    \begin{itemize}
      \tightlist
    \item
      \space\space Approximate \(\theta\) based on the random sample \(\textit{stat}\)(T(\(\theta\),\text{ sample size})) \newline
    \item
      \space\space Compute the squared error between theta and theta approximation \newline
    \item
      \space\space Compute the mean squared error \(\text{MSE}\) \newline
    \end{itemize}
    5. Print results \& draw the error plot \newline\newline\newline

    \vspace{230px}
    \begin{tcolorbox}[breakable, size=fbox, boxrule=1pt, pad at break*=1mm,colback=cellbackground, colframe=cellborder]
\prompt{In}{incolor}{67}{\boxspacing}
\begin{Verbatim}[commandchars=\\\{\}]
\PY{k}{def} \PY{n+nf}{error}\PY{p}{(}\PY{n}{theta}\PY{p}{,}\PY{n}{stat}\PY{p}{,} \PY{n}{T}\PY{p}{)}\PY{p}{:}
    \PY{n+nb}{print}\PY{p}{(}\PY{l+s+s2}{\PYZdq{}}\PY{l+s+s2}{Theta =}\PY{l+s+s2}{\PYZdq{}}\PY{p}{,}\PY{n}{theta}\PY{p}{)}
    \PY{n+nb}{print}\PY{p}{(}\PY{l+s+s2}{\PYZdq{}}\PY{l+s+s2}{Theta estimator:}\PY{l+s+s2}{\PYZdq{}}\PY{p}{,}\PY{n}{stat}\PY{p}{)}
    \PY{n}{mse} \PY{o}{=} \PY{l+m+mi}{0}
    \PY{n}{error\PYZus{}arr}\PY{o}{=}\PY{p}{[}\PY{p}{]}
    \PY{n}{sample\PYZus{}sizes}\PY{o}{=}\PY{n+nb}{range}\PY{p}{(}\PY{l+m+mi}{1000}\PY{p}{,}\PY{l+m+mi}{101000}\PY{p}{,}\PY{l+m+mi}{1000}\PY{p}{)}
    \PY{k}{for} \PY{n}{sample\PYZus{}size} \PY{o+ow}{in} \PY{n}{sample\PYZus{}sizes}\PY{p}{:}
        \PY{n}{theta\PYZus{}appr} \PY{o}{=} \PY{n}{stat}\PY{p}{(}\PY{n}{T}\PY{p}{(}\PY{n}{theta}\PY{p}{,}\PY{n}{sample\PYZus{}size}\PY{p}{)}\PY{p}{)}
        \PY{n}{sqe}\PY{o}{=}\PY{p}{(}\PY{n}{theta}\PY{o}{\PYZhy{}}\PY{n}{theta\PYZus{}appr}\PY{p}{)}\PY{o}{*}\PY{o}{*}\PY{l+m+mi}{2}
        \PY{n}{mse}\PY{o}{+}\PY{o}{=}\PY{n}{sqe}
        \PY{n}{error\PYZus{}arr}\PY{o}{.}\PY{n}{append}\PY{p}{(}\PY{n}{sqe}\PY{p}{)}
    \PY{n}{mse}\PY{o}{/}\PY{o}{=}\PY{n+nb}{len}\PY{p}{(}\PY{n}{sample\PYZus{}sizes}\PY{p}{)}
    \PY{n+nb}{print}\PY{p}{(}\PY{l+s+s2}{\PYZdq{}}\PY{l+s+s2}{MSE:}\PY{l+s+s2}{\PYZdq{}}\PY{p}{,}\PY{n}{mse}\PY{p}{)}
    \PY{n}{plt}\PY{o}{.}\PY{n}{plot}\PY{p}{(}\PY{n}{sample\PYZus{}sizes}\PY{p}{,}\PY{n}{error\PYZus{}arr}\PY{p}{,} \PY{n}{color}\PY{o}{=}\PY{l+s+s2}{\PYZdq{}}\PY{l+s+s2}{b}\PY{l+s+s2}{\PYZdq{}}\PY{p}{,} \PY{n}{label}\PY{o}{=}\PY{l+s+s2}{\PYZdq{}}\PY{l+s+s2}{squared error}\PY{l+s+s2}{\PYZdq{}}\PY{p}{)}
    \PY{n}{plt}\PY{o}{.}\PY{n}{axhline}\PY{p}{(}\PY{n}{mse}\PY{p}{,}\PY{n}{color}\PY{o}{=}\PY{l+s+s2}{\PYZdq{}}\PY{l+s+s2}{r}\PY{l+s+s2}{\PYZdq{}}\PY{p}{,} \PY{n}{label}\PY{o}{=}\PY{l+s+s2}{\PYZdq{}}\PY{l+s+s2}{mean squared error}\PY{l+s+s2}{\PYZdq{}}\PY{p}{)}
    \PY{k}{return} \PY{n}{mse}
\end{Verbatim}
\end{tcolorbox}

    \hypertarget{empirical-pdf-preview}{%
\subparagraph{Empirical PDF Preview}\label{empirical-pdf-preview}}

\begin{itemize}
\tightlist
\item
  A method \small\textbf{plt.hist()}  is used for plotting the
  empirical PDF
\item
  \small\textbf{density=True}  indicates that the histogram
  represents a probability density function
\end{itemize}

    \begin{tcolorbox}[breakable, size=fbox, boxrule=1pt, pad at break*=1mm,colback=cellbackground, colframe=cellborder]
\prompt{In}{incolor}{68}{\boxspacing}
\begin{Verbatim}[commandchars=\\\{\}]
\PY{n}{plt}\PY{o}{.}\PY{n}{hist}\PY{p}{(}\PY{n}{T}\PY{p}{(}\PY{n}{theta}\PY{p}{,}\PY{l+m+mi}{10000}\PY{p}{)}\PY{p}{,}\PY{n}{bins}\PY{o}{=}\PY{l+m+mi}{50}\PY{p}{,}\PY{n}{density}\PY{o}{=}\PY{k+kc}{True}\PY{p}{)}
\PY{n}{plt}\PY{o}{.}\PY{n}{title}\PY{p}{(}\PY{l+s+s2}{\PYZdq{}}\PY{l+s+s2}{Empirical PDF function of sample X}\PY{l+s+s2}{\PYZdq{}}\PY{p}{)}
\PY{n}{plt}\PY{o}{.}\PY{n}{gcf}\PY{p}{(}\PY{p}{)}\PY{o}{.}\PY{n}{set\PYZus{}size\PYZus{}inches}\PY{p}{(}\PY{l+m+mi}{3}\PY{p}{,}\PY{l+m+mi}{2}\PY{p}{)}
\PY{n}{plt}\PY{o}{.}\PY{n}{show}\PY{p}{(}\PY{p}{)}
\end{Verbatim}
\end{tcolorbox}

    \begin{center}
    \adjustimage{max size={0.9\linewidth}{0.9\paperheight}}{Lab2_files/Lab2_18_0.png}
    \end{center}
    { \hspace*{\fill} \\}
    
    \vspace{80px}
    \hypertarget{theta-estimation}{%
\subparagraph{Theta estimation}\label{theta-estimation}}

\begin{itemize}
\tightlist
\item
  In the example below, \(\textbf{ error() }\) function is used
  for computing the \(\theta\) estimation MSE (and the SE for each sample size):
\end{itemize}

\vspace{20px}
    \begin{tcolorbox}[breakable, size=fbox, boxrule=1pt, pad at break*=1mm,colback=cellbackground, colframe=cellborder]
\prompt{In}{incolor}{71}{\boxspacing}
\begin{Verbatim}[commandchars=\\\{\}]
\PY{n}{plt}\PY{o}{.}\PY{n}{gcf}\PY{p}{(}\PY{p}{)}\PY{o}{.}\PY{n}{set\PYZus{}size\PYZus{}inches}\PY{p}{(}\PY{l+m+mi}{4}\PY{p}{,}\PY{l+m+mf}{2.5}\PY{p}{)}
\PY{n}{plt}\PY{o}{.}\PY{n}{xlabel}\PY{p}{(}\PY{l+s+s2}{\PYZdq{}}\PY{l+s+s2}{sample size}\PY{l+s+s2}{\PYZdq{}}\PY{p}{)}
\PY{n}{plt}\PY{o}{.}\PY{n}{ylabel}\PY{p}{(}\PY{l+s+s2}{\PYZdq{}}\PY{l+s+s2}{squared error}\PY{l+s+s2}{\PYZdq{}}\PY{p}{)}
\PY{n}{mse} \PY{o}{=} \PY{n}{error}\PY{p}{(}\PY{n}{theta}\PY{p}{,}\PY{n}{Max}\PY{p}{,} \PY{n}{T}\PY{p}{)}
\PY{n}{plt}\PY{o}{.}\PY{n}{legend}\PY{p}{(}\PY{n}{loc}\PY{o}{=}\PY{l+m+mi}{1}\PY{p}{)}
\PY{n}{plt}\PY{o}{.}\PY{n}{title}\PY{p}{(}\PY{l+s+s2}{\PYZdq{}}\PY{l+s+s2}{Theta estimation (the estimator is sufficient), error plot}\PY{l+s+s2}{\PYZdq{}}\PY{p}{)}
\PY{n}{plt}\PY{o}{.}\PY{n}{show}\PY{p}{(}\PY{p}{)}
\PY{k}{if} \PY{p}{(}\PY{n}{mse}\PY{o}{\PYZlt{}}\PY{n}{Lambda}\PY{p}{)}\PY{p}{:}
    \PY{n}{success}\PY{p}{(}\PY{l+s+s2}{\PYZdq{}}\PY{l+s+s2}{The estimator }\PY{l+s+si}{\PYZpc{}s}\PY{l+s+s2}{ is sufficient.}\PY{l+s+s2}{\PYZdq{}}\PY{o}{\PYZpc{}}\PY{k}{Max})
\PY{k}{else}\PY{p}{:}
    \PY{n}{fail}\PY{p}{(}\PY{l+s+s2}{\PYZdq{}}\PY{l+s+s2}{The estimator }\PY{l+s+si}{\PYZpc{}s}\PY{l+s+s2}{ is not sufficient.}\PY{l+s+s2}{\PYZdq{}}\PY{o}{\PYZpc{}}\PY{k}{Max})
\end{Verbatim}
\end{tcolorbox}

    \begin{Verbatim}[commandchars=\\\{\}]
Theta = 10
Theta estimator: <function Max at 0x000001F5E2675510>
MSE: 5.81649238563254e-06
    \end{Verbatim}

    \begin{center}
    \adjustimage{max size={0.9\linewidth}{0.9\paperheight}}{Lab2_files/Lab2_20_1.png}
    \end{center}
    { \hspace*{\fill} \\}
    
    \begin{Verbatim}[commandchars=\\\{\}]
\textcolor{ansi-green-intense}{The estimator <function Max at 0x000001F5E2675510> is sufficient.}
    \end{Verbatim}

    \begin{itemize}

\vspace{120px}
\tightlist
\item
  \(\text{To illustrate the difference between consistent and incinsistent estimators, } \theta \newline \text{ is estimated with a different statistics function }- \small{\textbf{Mean(X)}}, \text{which is not consistent}\)
\end{itemize}
\vspace{20px}
    \begin{tcolorbox}[breakable, size=fbox, boxrule=1pt, pad at break*=1mm,colback=cellbackground, colframe=cellborder]
\prompt{In}{incolor}{72}{\boxspacing}
\begin{Verbatim}[commandchars=\\\{\}]
\PY{n}{plt}\PY{o}{.}\PY{n}{gcf}\PY{p}{(}\PY{p}{)}\PY{o}{.}\PY{n}{set\PYZus{}size\PYZus{}inches}\PY{p}{(}\PY{l+m+mi}{4}\PY{p}{,}\PY{l+m+mf}{2.5}\PY{p}{)}
\PY{n}{plt}\PY{o}{.}\PY{n}{xlabel}\PY{p}{(}\PY{l+s+s2}{\PYZdq{}}\PY{l+s+s2}{sample size}\PY{l+s+s2}{\PYZdq{}}\PY{p}{)}
\PY{n}{plt}\PY{o}{.}\PY{n}{ylabel}\PY{p}{(}\PY{l+s+s2}{\PYZdq{}}\PY{l+s+s2}{squared error}\PY{l+s+s2}{\PYZdq{}}\PY{p}{)}
\PY{n}{mse} \PY{o}{=} \PY{n}{error}\PY{p}{(}\PY{n}{theta}\PY{p}{,}\PY{n}{Mean}\PY{p}{,} \PY{n}{T}\PY{p}{)}
\PY{n}{plt}\PY{o}{.}\PY{n}{legend}\PY{p}{(}\PY{n}{loc}\PY{o}{=}\PY{l+m+mi}{4}\PY{p}{)}
\PY{n}{plt}\PY{o}{.}\PY{n}{title}\PY{p}{(}\PY{l+s+s2}{\PYZdq{}}\PY{l+s+s2}{Theta estimation (estimator is not sufficient), error plot}\PY{l+s+s2}{\PYZdq{}}\PY{p}{)}
\PY{n}{plt}\PY{o}{.}\PY{n}{show}\PY{p}{(}\PY{p}{)}
\PY{k}{if} \PY{p}{(}\PY{n}{mse}\PY{o}{\PYZlt{}}\PY{n}{Lambda}\PY{p}{)}\PY{p}{:}
    \PY{n}{success}\PY{p}{(}\PY{l+s+s2}{\PYZdq{}}\PY{l+s+s2}{The estimator }\PY{l+s+si}{\PYZpc{}s}\PY{l+s+s2}{ is sufficient.}\PY{l+s+s2}{\PYZdq{}}\PY{o}{\PYZpc{}}\PY{k}{Mean})
\PY{k}{else}\PY{p}{:}
    \PY{n}{fail}\PY{p}{(}\PY{l+s+s2}{\PYZdq{}}\PY{l+s+s2}{The estimator }\PY{l+s+si}{\PYZpc{}s}\PY{l+s+s2}{ is not sufficient.}\PY{l+s+s2}{\PYZdq{}}\PY{o}{\PYZpc{}}\PY{k}{Mean})
\end{Verbatim}
\end{tcolorbox}

    \begin{Verbatim}[commandchars=\\\{\}]
Theta = 10
Theta estimator: <function Mean at 0x000001F5E1EF15A0>
MSE: 25.000415469126352
    \end{Verbatim}

    \begin{center}
    \adjustimage{max size={0.9\linewidth}{0.9\paperheight}}{Lab2_files/Lab2_22_1.png}
    \end{center}
    { \hspace*{\fill} \\}
    
    \begin{Verbatim}[commandchars=\\\{\}]
\textcolor{ansi-red-intense}{The estimator <function Mean at 0x000001F5E1EF15A0> is not sufficient.}
    \end{Verbatim}

    \vspace{120px}
    \hypertarget{conclusions}{%
\subparagraph{7. Conclusions}\label{conclusions}}

\begin{enumerate}
\def\labelenumi{\arabic{enumi}.}
\vspace{15px}
\tightlist
\item 
  Sample distribution analysis conclusions:
\end{enumerate}

\begin{itemize}
\tightlist
\item
  \(\space X \equiv \textit{Unif }[0, \theta]\)

  \end{itemize}

\begin{enumerate}
\def\labelenumi{\arabic{enumi}.}
\setcounter{enumi}{1}
\tightlist
\vspace{10px}
  \item
  \(\text{H0}\):
  \(\space\textbf{Max(X)}\equiv X_{n:n}\text{ is a sufficient statistics for the parameter} \newline \text{(Max(X) is asymptotically normal for the parameter theta, because the}\newline\text{ SE (standard error)} \xrightarrow{} 0, n\xrightarrow{}\infty\newline\text{as shown using the function error())}\)

  \vspace{10px}
\item
  It is not possible to reject the \(\small{\textbf{Kolmogorov-Smirnov}}\) test H0, which
  states that the empirical CDF of X does not come from a family of Uniform distribution:
  \(\textit{Unif }[0, \theta]\), p-value=0.05
  \vspace{15px}

  \item
  Random sample generator \(T(\theta, n)\) generates simple random
  samples, that have distribution equivalent to \(\textit{Unif }[0, \theta]\)
  
\vspace{15px}
\item
  Statistical Model:
\end{enumerate}

\begin{itemize}
\tightlist
\item
  \((X,\{P_{\theta}, \theta \in \Theta\})\)

  \begin{itemize}
  \tightlist
  \vspace{10px}
  \item
    X = \([ 0,\space \theta]\)
    \vspace{5px}
  \item
    \(\theta = 10\); \(\Theta = \mathbb{R}\)
    \vspace{5px}
  \item
    \(P_{\theta} \equiv \textit{ Unif}\space(0, \theta)\)
  \end{itemize}
\end{itemize}



    % Add a bibliography block to the postdoc
    
    
    
\end{document}
